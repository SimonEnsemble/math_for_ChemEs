\documentclass[addpoints]{exam}
\usepackage{cmbright}
\usepackage{amsmath}
\usepackage{amssymb}
\usepackage{mhchem}

\header{CHE 525: math for CBEE's}{homework \#1}{fall 2025}

\begin{document}

\begin{questions}
	\subsection*{Linear systems of equations in chemical engineering}
 	\question a bioreactor is a vessel in which biological conversion is carried out involving enzymes, micro-organisms, and/or animal and plant cells.
	in the anaerobic fermentation of grain, the yeast \emph{Saccharomyces cerevisiae} digests glucose (\ce{C6H12O6}) from plants to form the products ethanol (\ce{C2H5OH}) and propenoic acid (\ce{C2H3CO2H}) by the following overall reactions:
	
	\hspace{2em} (rxn \#1): \ce{C6H12O6 -> 2C2H5OH + 2 CO2}
	
	\hspace{2em} (rxn \#2): \ce{C6H12O6 -> 2C2H3CO2H + 2H2O}
	
	in a batch process, a tank is charged with 4000\,kg of a 12\% solution of glucose in water. after fermentation, 120\,kg of \ce{CO2} are produced and 90\,kg of unreacted glucose remains in the broth. 
	our goal is to determine the mass of ethanol and propenoic acid in the broth at the end of the fermentation process.
	we assume none of the glucose is assimilated into the bacteria.
	
	\begin{parts}
		\part write a list defining the variables that represent the pertinent unknowns in this problem. e.g., ``$m_{\ce{C2H5OH}}$ [kg]: the mass of ethanol in the bioreactor at the end of the fermentation process.''. note: you should have five unknowns.
		\part apply a mass balance to arrive at a system of five equations for the five unknowns.
		\part formulate the system of equations as a matrix-vector equation $\mathbb{A}x=\mathbf{b}$.
		\part use Julia to solve the linear system of equations via building the coefficient matrix $\mathbb{A}$ and right-hand size vector $\mathbf{b}$.
		\part use \texttt{UnicodePlots.jl} or \texttt{CairoMakie.jl} to draw a bar plot showing the \emph{mass fraction} of glucose, ethanol, and propenoic acid in the bioreactor broth at the end of the fermentation process (note: \ce{CO2} of course is in the gas phase.)
	\end{parts}
	 
	
	(\emph{source}: ``Basic principles and calculations in chemical engineering'' by Himmelblau and Riggs, 7th Ed., 2004)
	
	
	 \question a winery in Oregon produces a Bordeaux-style red wine blend composed of a post-fermentation mixture of Cabernet Sauvignon, Merlot, and Cabernet Franc wines. 
	 we have many barrels of our delicious and popular Bordeaux-style red wine blend we made last year.
	 however, we forgot the make-up of the blend in terms of the three components.
	to estimate the volume percent of the different components, we measured the alcohol and sugar content of: (i) the pure Cabernet Sauvignon, Merlot, and Cabernet Franc from last year and (ii) our blend. use this data to set up then solve (in Julia) a system of linear equations to estimate the volume \% Cabernet Sauvignon, Merlot, and Cabernet Franc in the blend. plot the volume percents as a pie chart or bar chart using \texttt{UnicodePlots.jl} or \texttt{CairoMakie.jl}.
	  
	 \begin{table}[h!]
	 \centering
	  \begin{tabular}{c c c c }
		\hline
		wine variety & alcohol [g/L] & sugar [g/L]  \\
		\hline
		Cabernet Sauvignon &  108 & 2.2 \\
		Merlot &  114 & 3.0 \\
		Cabernet Franc & 112 & 0.1 \\
		\hline
		blend & 112.1 & 1.43 
	\end{tabular}
	\end{table}
	
	\subsection*{Vectors}
	\question describe geometrically (line, plane, or all of $\mathbb{R}^3$) all linear combinations of:
	\begin{parts}
		\part $\begin{bmatrix}  1 \\ 2 \\3 \end{bmatrix}$ and $\begin{bmatrix}  3\\ 6\\9 \end{bmatrix}$
		\part $\begin{bmatrix}  1 \\ 0 \\0 \end{bmatrix}$ and $\begin{bmatrix}  0\\ 2\\3 \end{bmatrix}$
		\part $\begin{bmatrix}  2 \\ 0 \\0 \end{bmatrix}$,  $\begin{bmatrix}  0\\ 2 \\2 \end{bmatrix}$, and $\begin{bmatrix}  2\\ 2\\3 \end{bmatrix}$
		\part $\begin{bmatrix}  1\\ 2 \\3 \end{bmatrix}$,  $\begin{bmatrix}  -3\\ 1 \\-2 \end{bmatrix}$, and $\begin{bmatrix}  2\\ -3\\-1 \end{bmatrix}$
	\end{parts}
	
	\question write two equations for $c$ and $d$ so that the linear combination $c\mathbf{v}+d\mathbf{w}=\mathbf{b}$, with:
	\begin{equation*}
		\mathbf{v} = \begin{bmatrix} 2 \\ -1 \end{bmatrix}, \mathbf{w} = \begin{bmatrix} -1 \\ 2 \end{bmatrix}, \mathbf{b} = \begin{bmatrix} 0 \\ 1 \end{bmatrix}.
	\end{equation*}
	
	\question draw 
	$\mathbf{v} = \begin{bmatrix} 4 \\ 1 \end{bmatrix}$ and $\mathbf{w} = \begin{bmatrix} -2 \\ 2 \end{bmatrix}$, $\mathbf{v}+\mathbf{w}$, and $\mathbf{v}-\mathbf{w}$ in the 2D plane.
	
	\question find two vectors $\mathbf{v}$ and $\mathbf{w}$ such that $\mathbf{v}+\mathbf{w}=\begin{bmatrix}  4 \\ 5 \\6 \end{bmatrix}$ and $\mathbf{v}-\mathbf{w}=\begin{bmatrix}  2 \\ 5 \\8 \end{bmatrix}$.
	
	\question find two different combinations of $\mathbf{u}=\begin{bmatrix}  1\\3 \end{bmatrix}$, $\mathbf{v}=\begin{bmatrix}  2\\7 \end{bmatrix}$, and $\mathbf{w}=\begin{bmatrix}  1\\5 \end{bmatrix}$ that produce  $\mathbf{b}=\begin{bmatrix}  0\\1\end{bmatrix}$.
	
	\question find a unit vector $\mathbf{u}$ in the direction of $\mathbf{v}= \begin{bmatrix} 1 \\ 3 \end{bmatrix}$. find a unit vector $\mathbf{u}_\perp$ that is perpendicular to $\mathbf{v}$.
	
	\question compute the angle between:
	\begin{equation*}
		\mathbf{v} = \begin{bmatrix} 1 \\ \sqrt{3} \end{bmatrix}, \mathbf{w} = \begin{bmatrix} -1 \\ \sqrt{3} \end{bmatrix}
	\end{equation*}
	
	\question find nonzero vectors $\mathbf{v}$ and $\mathbf{w}$ that are perpendicular to $\begin{bmatrix}  1\\0\\1 \end{bmatrix}$ and each other.
	
	\question if the dot product $\mathbf{v}\cdot \mathbf{w}$ is negative, what does this say about the two vectors? sketch two examples where $\mathbf{v}\cdot \mathbf{w} < 0$ in the 2D plane.
	 
	 \subsection*{Matrix times vectors}
	 
	 \question solve these three equations for $y_1, y_2, y_3$ in terms of $c_1, c_2, c_3$:
	 \begin{equation*}
	  \mathbb{S}\mathbf{y}=\mathbf{c} \hspace{5em} 
	  \begin{bmatrix} 1 & 0 & 0 \\ 1 & 1 & 0\\ 1 & 1 & 1 \end{bmatrix} 
	  \begin{bmatrix} y_1 \\ y_2 \\ y_3 \end{bmatrix} =
	  \begin{bmatrix} c_1 \\ c_2 \\ c_3 \end{bmatrix}
	 \end{equation*}
	 write the solution as a vector $\mathbf{y}$. then write the solution as a matrix $\mathbb{S}^{-1}$ times the vector $\mathbf{c}$. are the columns of $\mathbf{S}$ dependent or independent?
	 
	 \question find a combination $x_1\mathbf{w}_1+x_2\mathbf{w}_2+x_3\mathbf{w}_3$ that gives the zero vector with $x_1=1$. 
	 \begin{equation*}
		\mathbf{w_1} = \begin{bmatrix} 1 \\ 2 \\3 \end{bmatrix}, \mathbf{w_2} = \begin{bmatrix} 4 \\ 5 \\6 \end{bmatrix}, \mathbf{w_3} = \begin{bmatrix} 7 \\ 8 \\9 \end{bmatrix}, 
	\end{equation*}
	 those vectors are \fillin (independent or dependent?). the matrix $\mathbb{W}$ with those columns is \fillin (invertible or singular?). (I suggest you plot these vectors in desmos to visualize them and see for yourself.)
	
	 
	 \question what number $c$ gives dependent columns so that the matrix is singular?
	 \begin{equation*}
	  \begin{bmatrix} 1 & 1 & 0 \\ 3 & 2 & 1\\ 7 & 4 & c \end{bmatrix} 
	 \end{equation*}
	 
 \end{questions}
\end{document}